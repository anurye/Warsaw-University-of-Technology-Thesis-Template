% -*- mode:LaTex; mode:visual-line; mode:flyspell; fill-column:75-*-

% Define any helpful macros here.

% Parentheses, braces, brackets.
\newcommand*{\parens}[1]{\left( #1 \right)}
\newcommand*{\paren}[1]{\left( #1 \right)}
\newcommand*{\braces}[1]{ \{ #1 \} }
\newcommand*{\brackets}[1]{ \left[ #1 \right] }

% Math stuff.
\DeclareMathOperator*{\argmin}{\arg\!\min}
\DeclareMathOperator*{\argmax}{\arg\!\max}

% Default table and figure dimensions.
\newcommand{\defaultTableWidth}{0.9 \textwidth}
\newcommand{\defaultFigWidth}{0.65}
\newcommand{\defaultAxisWidth}{0.7\textwidth}
\newcommand{\defaultAxisHeight}{0.5\textwidth}

% Footnote for an entire chapter -- no symbol, just text at the bottom.
\newcommand{\chapternote}[1]{{%
  \let\thempfn\relax% Remove footnote number printing mechanism
  \footnotetext[0]{\emph{#1}}% Print footnote text
}}

%% This will add the subgroups, e.g, for nomenclature
%----------------------------------------------
\usepackage{etoolbox}
\renewcommand\nomgroup[1]{%
  \item[\bfseries
  \ifstrequal{#1}{A}{Physics Constants}{%
  \ifstrequal{#1}{B}{Number Sets}{%
  \ifstrequal{#1}{C}{Other Symbols}{}}}%
]}
%----------------------------------------------
% -----------------------------------------
%% This will add the units
%----------------------------------------------
\newcommand{\nomunit}[1]{%
\renewcommand{\nomentryend}{\hspace*{\fill}#1}}
%----------------------------------------------
